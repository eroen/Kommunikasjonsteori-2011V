%!TEX TS−program = xetex
%!TEX encoding = UTF−8 Unicode
\documentclass[a4paper, twoside, fleqn]{scrartcl}
\usepackage{amssymb}
\usepackage{amsbsy}
\usepackage{amsmath}
\usepackage{amsthm}
\usepackage{undertilde}
\usepackage{xunicode}
\usepackage{fontspec}
\usepackage{xltxtra}
\usepackage{polyglossia}
\setmainlanguage{english}
\setromanfont[Mapping=tex-text]{Linux Libertine O}
\setsansfont[Mapping=tex-text]{Linux Biolinum O}
\setmonofont[Mapping=tex-text, Scale=0.8]{DejaVu Sans Mono}
% \setmonofont[Mapping=tex-text, Scale=0.9]{Courier New}
%\setromanfont[Mapping=tex-text]{Times New Roman}
%\setromanfont[Mapping=tex-text]{Arial}
\usepackage{units}
\usepackage{graphicx}
\usepackage{float}
\usepackage{color}
% \usepackage{transparent}
\usepackage{listings}

\renewcommand{\labelenumi}{\alph{enumi})}
\renewcommand{\theenumi}{\alph{enumi})}
\renewcommand{\labelenumii}{\roman{enumii})}
%\renewcommand{\vec}[1]{\ensuremath{\mathbf{#1}}} % for vectors
\renewcommand{\vec}[1]{\ensuremath{\boldsymbol{#1}}} % for vectors
\newcommand{\mat}[1]{\ensuremath{\utilde{#1}}} % for matrices
\newcommand{\curl}[1]{\ensuremath{\mathbf{\nabla} \times #1}} % 
\newcommand{\abs}[1]{\ensuremath{\left| #1 \right|}} % 
\newcommand{\conv}[0]{\ensuremath{\ast}} % 
\newcommand{\E}[0]{\ensuremath{\text{E}}} % 
\newcommand{\erf}[0]{\ensuremath{\text{erf}}} % 
\newcommand{\re}[1]{\ensuremath{\text{Re}\left\{ #1 \right\}}} % 
\renewcommand{\exp}[1]{\ensuremath{\text{e}^{ #1 }}} % 
% \def\svgwidth{5cm}
% \def\unitlength{10cm}
\begin{document}
\title{Exercise 2}
\subject{TTT4115 Kommunikasjonsteori}
\author{Erik Moen}
\maketitle

\section*{Problem 1}
  \label{1}
  \begin{enumerate}
    \item 
      \label{1a}
      \begin{align}
	R_y(\tau) &= y(t) \conv y(\tau - t) \\
	&= h(t) \conv x(t) \conv h(\tau - t) \conv x(\tau - t) \\
	&= x(t) \conv x(\tau - t) \conv h(t) \conv h(\tau - t) \\
	&= R_x(\tau) \conv \left( \int_{-\infty}^{\infty} h(t) h(t + \tau) \, dt \right)
      \end{align}
    
    \item
      \label{1b}
      The rightmost part of the righthand side of the result in \ref{1a} is the autocorrelation of the filter transfer function and gives the leftmost part of the righthand side of the equation in \ref{1b} as a power spectrum when fourier transformed.
      
      The leftmost part of the righthand side of the result in \ref{1a} obviously gives the input power spectral density when fourier transformed.
      
      Lastly, a convolution is replaced with a multiplication when fourier transform is performed.
      
    \item
      \label{1c}
      \begin{align}
        \sigma_y^2 &= E\{ y^2 \} - E\{y\}^2 = E\{ y^2 \} \\
        &= R_y(0) \\
        &= R_x(0) \conv \left( \int_{-\infty}^{\infty} h(t) h(t) \, dt \right) \\
        &= R_x(0) \cdot \left( \int_{-\infty}^{\infty} h(t)^2 \, dt \right)
      \end{align}
      
      \begin{align}
        \sigma_y^2 &= \int_0^\infty S_y(f) \, df \\
        &= \int_0^\infty \abs{H(f)}^2 S_x(f) \, df
      \end{align}


    \item
      \label{1d}
      X having unit variance gives:
      \begin{align}
        \sigma_x^2 &= 1 \\
        R_x(0) &= 1 
      \end{align}
      
      Remembering \ref{1c} we have:
      \begin{align}
        \sigma_y^2 &= R_x(0) \cdot \left( \int_{-\infty}^{\infty} h(t)^2 \, dt \right) \\
        \sigma_y^2 &= \int_{-\infty}^{\infty} h(t)^2 \, dt 
      \end{align}
      $h(t)$ is still real, so the absolute value operator makes no significant difference to the expression.
  \end{enumerate}
  

\section*{Problem 2}
  \label{2}
  \begin{enumerate}
    \item 
      \label{2a}
      \begin{gather}
        D(f) = X(f) - G(f) \big[ N(f) + H(f) X(f) \big]
      \end{gather}
      
    \item
      \label{2b}
      We remember that X(f) is independent of N(f), thus the cross-correlation of the two is zero.
      \begin{align}
        S_D(f) &= D(f) \cdot D^*(f) \\
        \begin{split}
          = S_X(f) + \abs{G(f)}^2 \big[ \abs{H(f)}^2 S_X(f) + S_N(f) + 2 \re{H(f) X(f) N^*(f)} \big] \\
          - 2 \re{X^*(f) G(f) \big[ H(f) X(f) + N(f) \big] }
        \end{split} \\
        \begin{split}
	  = S_X(f) + \abs{G(f)}^2 \big[ \abs{H(f)}^2 S_X(f) + S_N(f) \big] \\ 
	  - 2 \re{G(f) \big[ H(f) S_X(f) + X^*(f) N(f) \big] }
        \end{split} \\
        &= S_X(f) + \abs{G(f)}^2 \big[ \abs{H(f)}^2 S_X(f) + S_N(f) \big] - 2 \re{ G(f) H(f) } S_X(f)
      \end{align}
      This can of course be simplified further by assuming one or more of $G(f)$ and $H(f)$ are real.
      
    \item
      \label{2c}
      \begin{gather}
        \abs{G(f)}^2 = \frac{ \abs{H(f)}^2 S_X^2(f) }{ \abs{H(f)}^4 S_X^2(f) + S_N^2(f) + \abs{H(f)}^2 S_X(f) S_N(f) } \\
        = \frac{ \abs{H(f)}^2 S_X^2(f) }{ \abs{H(f)}^4 S_X^2(f) + S_N^2(f)} \\
        \begin{split}
	  S_D(f) = S_X(f) + \frac{ \abs{H(f)}^2 S_X^2(f) }{ \abs{H(f)}^4 S_X^2(f) + S_N^2(f)} ( \abs{H(f)}^2 S_X(f) + S_N(f) ) \\
	  - 2 \re{ \frac{ H^*(f) S_X(f) }{ \abs{H(f)}^2 S_X(f) + S_N(f) } H(f) S_X(f) }
	\end{split} \\
	\begin{split}
	  = S_X(f) + \frac{ \abs{H(f)}^2 S_X^2(f) }{ \abs{H(f)}^4 S_X^2(f) + S_N^2(f)} ( \abs{H(f)}^2 S_X(f) + S_N(f) ) \\
	  - 2 \frac{ \abs{H(f)}^2 S_X^2(f) }{ \abs{H(f)}^2 S_X(f) + S_N(f) }
	\end{split} \\
	\begin{split}
	  = S_X(f) + \frac{ \abs{H(f)}^2 S_X^2(f) \abs{H(f)}^2 S_X(f) + \abs{H(f)}^2 S_X^2(f) S_N(f) }{ \abs{H(f)}^4 S_X^2(f) + S_N^2(f)}  \\
	  - 2 \frac{ \abs{H(f)}^2 S_X^2(f) }{ \abs{H(f)}^2 S_X(f) + S_N(f) }
	\end{split} \\
	\begin{split}
	  = S_X(f) + \frac{ \abs{H(f)}^4 S_X^3(f) }{ \abs{H(f)}^4 S_X^2(f) + S_N^2(f)} %
	  - 2 \frac{ \abs{H(f)}^2 S_X^2(f) }{ \abs{H(f)}^2 S_X(f) + S_N(f) }
	\end{split}
      \end{gather}
      
    \item
      \label{2d}
      \begin{itemize}
        \item 
	  \begin{gather}
	    S_N(f) = \abs{\frac{N_0} 2}^2 = \frac{ \abs{N_0}^2} 4 \\
	    S_D= S_X(f) + \frac{ \abs{A}^4 S_X^3(f) }{ \abs{A}^4 S_X^2(f) + \frac{ \abs{N_0}^4} {16} } - 2 \frac{ \abs{A}^2 S_X^2(f) }{ \abs{A}^2 S_X(f) + \frac{ \abs{N_0}^2} 4 }
	  \end{gather}
	
	\item
	  \begin{gather}
% 	    \begin{split}
	      S_D= S_X(f) + \frac{ \abs{\frac{B}{\sqrt{S_X(f)}}}^4 S_X^3(f) }{ \abs{\frac{B}{\sqrt{S_X(f)}}}^4 S_X^2(f) + \frac{ \abs{N_0}^4} {16} } - 2 \frac{ \abs{\frac{B}{\sqrt{S_X(f)}}}^2 S_X^2(f) }{ \abs{\frac{B}{\sqrt{S_X(f)}}}^2 S_X(f) + \frac{ \abs{N_0}^2} 4 } \\
	      %
	      S_D= S_X(f) + \frac{ \frac{\abs{B}^4}{S_X^2(f)} S_X^3(f) }{ \frac{\abs{B}^4}{S_X^2(f)} S_X^2(f) + \frac{ \abs{N_0}^4} {16} } - 2 \frac{ \frac{\abs{B}^2}{S_X(f)} S_X^2(f) }{ \frac{\abs{B}^2}{S_X(f)} S_X(f) + \frac{ \abs{N_0}^2} 4 } \\
	      %
	      S_D= S_X(f) + \frac{ {\abs{B}^4} S_X(f) }{ {\abs{B}^4} + \frac{ \abs{N_0}^4} {16} } - 2 \frac{ {\abs{B}^2} S_X(f) }{ {\abs{B}^2} + \frac{ \abs{N_0}^2} 4 } \\
	      %
	      S_D= S_X(f) \left( 1 + \frac{ {\abs{B}^4} }{ {\abs{B}^4} + \frac{ \abs{N_0}^4} {16} } - 2 \frac{ {\abs{B}^2} }{ {\abs{B}^2} + \frac{ \abs{N_0}^2} 4 } \right)
% 	    \end{split}
	  \end{gather}
	
	\item
	  \begin{gather}
	    A = \frac{C}{\sqrt[4]{S_X(f)}} \\
	    S_D= S_X(f) + \frac{ \abs{\frac{C}{\sqrt[4]{S_X(f)}}}^4 S_X^3(f) }{ \abs{\frac{C}{\sqrt[4]{S_X(f)}}}^4 S_X^2(f) + \frac{ \abs{N_0}^4} {16} } %
	    - 2 \frac{ \abs{\frac{C}{\sqrt[4]{S_X(f)}}}^2 S_X^2(f) }{ \abs{\frac{C}{\sqrt[4]{S_X(f)}}}^2 S_X(f) + \frac{ \abs{N_0}^2} 4 } \\
	    %
	    S_D= S_X(f) + \frac{ \frac{\abs{C}^4}{{S_X(f)}} S_X^3(f) }{ \frac{\abs{C}^4}{{S_X(f)}} S_X^2(f) + \frac{ \abs{N_0}^4} {16} } %
	    - 2 \frac{ \frac{\abs{C}^2}{\sqrt{S_X(f)}} S_X^2(f) }{ \frac{\abs{C}^2}{\sqrt{S_X(f)}} S_X(f) + \frac{ \abs{N_0}^2} 4 } \\
	    %
	    S_D= S_X(f) + \frac{ {\abs{C}^4} S_X^2(f) }{ {\abs{C}^4} S_X(f) + \frac{ \abs{N_0}^4} {16} } %
	    - 2 \frac{ {\abs{C}^2} S_X^{\frac 3 2}(f) }{ {\abs{C}^2} \sqrt{S_X(f)} + \frac{ \abs{N_0}^2} 4 }
	  \end{gather}

      \end{itemize}
    
    \item
      \label{2e}
    \item
      \label{2f}
    \item
      \label{2g}
      Sub-problems \ref{2e} -- \ref{2g} are not answered, due to time constraints.

  \end{enumerate}

\end{document}
